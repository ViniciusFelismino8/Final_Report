\chapter{Materiais e Métodos}
\label{chap:mat}

Neste capítulo será descrito cada desafio proposto pelos os orientadores e pesquisadores do CCRoSA durante o ano letivo e os materiais e métodos adotados para a concepção das soluções para cada um deles.

% Neste capítulo será descrito os materiais e os métodos adotados para a concepção das soluções para cada desafio proposto pelo os orientadores e pesquisadores do CCRoSA durante o ano letivo. 

\section{Desafio 2.0 - Concepção de um manipulador robótico.}
\label{sec:desafio2.0}

A princípio o objetivo deste desafio foi construir um manipulador robótico e através da identificação de marcadores visuais realizar a tarefa de acionamento de um painel elétrico. Devido aos efeitos da COVID-19, as atividades presenciais no laboratório tiveram que ser suspensas, com isso, afetou diretamente na construção física do manipulador. Portanto, a solução entregue para este desafio foi realizada apenas em ambiente simulado. 

Para concepção da solução foi formado um grupo que, além de mim, detinha mais três bolsistas do programa, entre eles, Jéssica Motta, Leonardo Mendes e Miguel Felipe. A modelagem 3D da estrutura física do manipulador foi feita no \textit{software OnShape}, o \textit{framework} de robótica e manipulação utilizado foi o \textit{Robot Operating System (ROS)} e \textit{MoveIt}, respectivamente. Por fim, a simulação da tarefa foi realizada no \textit{software Gazebo} com auxilio da ferramenta de visualização \textit{Rviz}.

\section{Desafio 2.5 - Simular a marcha e corrida de revezamento com o Darwin-OP}
\label{sec:desafio2.5}