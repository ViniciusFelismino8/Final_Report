\chapter{Introdução}
\label{chap:intro}

O presente documento visa agrupar todos os conteúdos abordados e mostrar os resultados das atividades de pesquisa, criação e seleção de soluções para todos os desafios propostos pelos os orientadores e pesquisadores do Centro de Competência em Robótica e Sistemas Autônomos (CCRoSA) do SENAI-CIMATEC. Todas estas atividades ocorram durante o processo do curso de formação em robótica e sistemas autônomos, oferecido pela instituição, que teve inicio no dia 9 de novembro de 2019 e um prazo de duração de 1 ano.



%--------- NEW SECTION ----------------------
\section{Objetivos}
\label{sec:obj}

O objetivo geral deste relatório é relatar como o curso foi estruturado, ou seja, será apresentado todas as atividades e conhecimentos adiquiridos durante todo o periodo de duração da formação. 


\subsection{Objetivos Específicos}
\label{ssec:objesp}

O objetivo específico deste documento é agrupar todas as soluções para os diversos desafios proposto. 

%--------- NEW SECTION ----------------------
\section{Justificativa}
\label{sec:justi}

Com os crecentes avanços técnológicos, principalmente na área da robótica, a instituição SENAI-CIMATEC observou que era necessário mão de obra qualificada neste campo. Partindo desta premissa e adotando-a como seu principal objetivo lançou o programa de formação ``Novos Talentos - Robótica e Sistemas Autônomos'' para buscar o aumento da oferta de pessoal qualificado, em especial para as competências de robótica e sistemas Autônomos. 


%--------- NEW SECTION ----------------------
\section{Organização do documento}
\label{section:organizacao}

Este documento apresenta $5$ capítulos e está estruturado da seguinte forma:

\begin{itemize}

  \item \textbf{Capítulo \ref{chap:intro} - Introdução}: Contextualiza o âmbito, no qual o programa proposto está inserido. Apresenta, portanto, os objetivos, justificativa e como este programa foi estruturado;
  % \item \textbf{Capítulo \ref{chap:fundteor} - Fundamentação Teórica}: XXX;
  \item \textbf{Capítulo \ref{chap:mat} - Desenvolvimento}: Contextualiza cada problema proposto pelos os orientadores e pesquisadores do CCRoSA. Além disso, será demonstrado os materiais e métodos utilizados para a solução dos mesmos;
  \item \textbf{Capítulo \ref{chap:met} - Metodologia}: Será apresentada a metologia adotada pelos os orientadores durante o programa de formação;  
  \item \textbf{Capítulo \ref{chap:result} - Resultados}: Será exibido os principais resultados obtidos com a resolução de cada desafio proposto;
  \item \textbf{Capítulo \ref{chap:conc} - Conclusão}: Apresenta a conclusão geral do programa além de apresentar uma conclusão específica para cada desafio.

\end{itemize}
