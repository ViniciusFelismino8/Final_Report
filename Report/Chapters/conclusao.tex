\chapter{Conclusão}
\label{chap:conc}

O presente relatório buscou apresentar, além da metodologia utilizada, todas as atividades realizadas durante o período vigente do programa de formação. Com isso, permitiu adquirir conhecimentos nas áreas da robótica e sistema autônomos, consideradas como ramos tecnológicos multidisciplinar, principalmente no que se refere a manipuladores robóticos e robótica móvel.  

Além disso, o programa de formação apresentou ferramentas importantes para o desenvolvimento de toda e qualquer pesquisa, dentre elas destacam-se: Trello, Element, GitHub, RStudio, Visual Code. O trello e o element são duas ferramentas importantes no quesito de gestão de projetos pois a primeira, além de simples de operar e contar com uma versão gratuita, possibilita resolver situações de fluxo de atividades e da gestão de equipes no seu projeto por contar com uma série de integrações prontas com outros softwares, já a segunda possibilita uma comunicação segura entre os integrantes do projeto pois utiliza o método de criptografia das mensagens.


% forma geral, o programa de formação proporcionou o desenvolvimento de conhecimentose habilidades requeridas nas áreas de robótica e sistemas autônomos. Os resultados deriva-dos dos projetos, foram expostas no cápitulo 3 onde envolveu um enorme aprendizado deplanejamento, execução e entrega de projetos. Este documento mostrou o desenvolvimentode um especialista no curso de formação em Robótica e Sistemas Autônomos, que foiformado com base nas ferramentas utilizadas para modelagem, simulação e construçãoreal desses sistemas, e que são usadas no mundo todo nessa área de Robótica e SistemasAutônomos, nas linguagens de programações fundamentais comoC++, PythoneR, sobrecomo os estudos estatísticos são aplicados para fazer análise dos projetos, e saber elaboraro planejamento, direcionar a execução e entregar os resultados aos clientes do projetospropostos.12


% O presente relatório tem o objetivo de apresentar os trabalhos desenvolvidos durante operiodo vingente da bolsa de estudos em robótica e sistemas autônomos. Esta oportunidadesurgiu através do programa “Novos Talentos - Robótica e Sistemas Autônomos” oferecidopelo SENAI-CIMATEC no ano de 2019 cujo o principal objetivo foi de desenvolver capitalhumano diferenciado no estado da Bahia, buscando o aumento da oferta de mão de obraqualificada nas competências supracitadas. Os trabalhos desenvolvidos durante o periodo doprograma permitiram atribuir conhecimentos nas áreas de robótica móvel e manipuladoresrobóticos e, consequentimente, proporcionaram a formação de um Especialista em Robóticae Sistemas Autônomos.Palavras-chave: Robótica, Sistemas Autônomos, Robótica Móvel, Manipuladores.i



