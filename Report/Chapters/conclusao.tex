\chapter{Conclusão}
\label{chap:conc}

O presente relatório buscou apresentar, além da metodologia utilizada, todas as atividades realizadas durante o período vigente do programa de formação. Com isso, permitiu adquirir conhecimentos nas áreas da robótica e sistema autônomos, consideradas como ramos tecnológicos multidisciplinar, principalmente no que se refere a manipuladores robóticos e robótica móvel.  

O programa de formação apresentou ferramentas importantes para o desenvolvimento de toda e qualquer pesquisa nestas áreas, dentre elas destacam-se: \textit{Trello}, \textit{Element}, \textit{GitHub}, \textit{RStudio}, \textit{Visual Code}. O \textit{trello} e o \textit{element} são duas ferramentas importantes no quesito de gestão de projetos pois a primeira, além de simples de operar e contar com uma versão gratuita, possibilita resolver situações de fluxo de atividades e da gestão de equipes no seu projeto por contar com uma série de integrações prontas com outros softwares, já a segunda possibilita uma comunicação segura entre os integrantes do projeto pois utiliza o método de criptografia das mensagens. O \textit{GitHub} é uma ferramenta com grande importância pois possibilita o versionamento do projeto. Por fim, o \textit{Visual Code} e \textit{RStudio} são duas ferramentas importantes para desenvolvimento de código em qualquer linguagem de programação, além de possuir diversas extensões e bibliotecas para facilitar no seu uso. É importante lembrar que o programa sempre buscou utilizar ferramentas de natureza \textit{open source}.

Além disso, todos os projetos de robótica foram desenvolvidos utilizando o \textit{framework} \textit{ROS}, pois forneceu as bibliotecas e ferramentas necessárias para auxiliar na concepção das funcionalidades do manipulador e da plataforma móvel. Além de que esta ferramenta possibilitou a simulação de todas as atividades proposta em cada desafio e com isso observar que a simulação pode ser totalmente diferente do mundo real.  

Durante o programa também foram desenvolvidas atividades que não foram citadas neste relatório mas que também foram importantes, entre elas, as apresentações individuais sobre artigos com temas propostos, liderança em projetos e alguns encontro para discutir sobre gestão de projetos. Todas estas permitiram criar e/ou aperfeiçoar habilidades nestas áreas.

De forma geral, o programa de formação ``Novos talentos - Robótica e Sistemas Autônomos'' oferecido pelo SENAI-CIMATEC em novembro de 2019 até novembro de 2020 permitiu por em prática todos os conceitos de robótica e sistemas autônomos vistos durante o ano letivo e isto só foi possível graças ao corpo docente e a infraestrutura do Centro de Competência em Robótica e Sistemas Autônomos (CCRoSA). Com isso podemos concluir que este programa, mesmo com as adversidades causadas pelo COVID-19, atendeu aos objetivos oferecidos em sua programação e assim proporcionou a formação de um Especialista em Robótica e Sistemas Autônomos. 




