\chapter{Resultados}
\label{chap:result}
Neste capítulo serão descritos os resultados que cada relatório desenvolvido para as possíveis soluções de cada desafio gerou durante o programa de formação. Além disso, será apresentado um trabalho extra que, impulsionado devido aos efeitos do COVID-19, foi desenvolvido por alguns bolsistas juntamente com os orientadores e pesquisadores do centro de competência além do seu potencial resultado. 

%--------- NEW SECTION ----------------------
\section{Resultado do resumo expandido ``Timon-HM''}
\label{sec:testu}

O relatório desenvolvido durante a concepção da solução do desafio 2.0, conforme visto na seção \ref{sec:desafio2.0}, gerou, além da participação, a submissão do resumo expandido ``MANIPULADOR ROBÓTICO- TIMON-HM'' dos autores Jéssica Motta, Leonardo Lima, Miguel Felipe e Vinicius Felismino no V Seminário de Pesquisa Científica e
Tecnológica (SAPCT) e IV Workshop de Integração e Capacitação em Processamento de Alto Desempenho (ICPAD). Os certificados de participação e submissão estão disponíveis nos anexos \ref{appen:sapct} e \ref{appen:resume_timon}, respectivamente.

\section{UGV MOCÓ: ROBÔ AUTÔNOMO TERRESTRE PARA DESINFECÇÃO DE AMBIENTES}
\label{sec:ugv}

Em virtude da pandemia criou-se a necessidade de desinfecção e higienização de ambientes para que pessoas possam transitar com mais segurança em locais cujo risco de contaminação é controlado. De modo a atender tal necessidade, propõe-se desenvolver uma plataforma robótica móvel capaz de percorrer ambientes de forma autônoma, empregando métodos
de desinfecção recomendados pela comunidade científica. 

Deste modo, o Apendice \ref{appen:ugv} apresenta a fase de concepção do projeto, detalhando as soluções atuais e o conceito proposto. Traz-se a especificação do sistema, modelo esquemático e a arquitetura geral do sistema. A equipe responsável pelo desenvolvimento deste projeto foi: Anderson Queiroz do Vale, Israel Cerqueira Motta Neto, Vinicius José Gomes de Araújo Felismino, Rebeca Tourinho Lima e Marco Antonio dos Reis.


\section{Artigo publicado PERSPECTIVES ON AUTONOMOUS UNMANNED GROUND
VEHICLES: A SURVEY}

Atualmente, os robôs móveis têm sido usados em diferentes ambientes e
aplicações, até mesmo para tarefas domésticas. O uso desses dispositivos pode aumentar a segurança e a produtividade, especialmente em operações industriais.

Partindo deste conceito e atrelado com a concepção da plataforma móvel,  visto na seção \ref{sec:ugv}, este artigo teve o objetivo de identificar gaps tecnológicos por meio de uma seleção atualizada
e por uma comparação de veículos terrestres não tripulados com rodas (UGVs). Com base em uma revisão sistemática e em uma classificação de robôs comercialmente disponíveis. Os resultados desta pesquisa pode ser visto no Apêndice \ref{appen:ugv_artigo}. 

Além disso, este artigo foi aceito para para publicação no
VI International Symposium on Innovation and Technology (SIINTEC), realizado no período de 21/10/2020 a 23/10/2020. O certificado de publicação pode ser visto no Apêndice \ref{appen:ugv_certificado}.