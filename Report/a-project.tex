% a-project.tex, v-1.0.3 marcoreis baseado no
% abntex2-modelo-trabalho-academico.tex, v-1.9.7 laurocesar
% Copyright 2012-2018 by abnTeX2 group at http://www.abntex.net.br/ 
% 
% This work consists of the files ........
% 
% -----------------------------------------------------------------------------
% Modelo para desenvolvimento de documentação de projetos acadêmicos
% (tese de doutorado, dissertação de mestrado e trabalhos de monografias em geral) 
% em conformidade com ABNT NBR 14724:2011: Informação e documentação. 
% -----------------------------------------------------------------------------
% Opções para a documentação
%
% Fancy page headings 
%\documentclass[fancyheadings, subook]{Classes/a-prj}
%\documentclass[fancyheadings, sureport]{Classes/a-prj}
%
% Fancy chapters and sections headings 
%\documentclass[fancychapter, subook]{Classes/a-prj}
%\documentclass[fancychapter, sureport]{Classes/a-prj}
%
% Fancy page , chapters and sections headings
%\documentclass[fancyheadings, fancychapter, subook]{Classes/a-prj}
\documentclass[fancyheadings, fancychapter, sureport]{Classes/a-prj}
%
% -----------------------------------------------------------------------------
% Alguns comandos para a fancy page headings)
%
% Page header line width
%\footlinewidth{value}
%
% Page footer line width
%\headlinewidth{value}
%
% Page header and footer line width
%\headingslinewidth{value}
%
% Page header and footer lines without text
%\headingslinesonly
%
% The default line width is 0.3pt.
% Set the value to 0pt to remove the page header and/or footer line
%
% -------------------------------------------------------------------------------
% Formato de figuras suportado
% -------------------------------------------------------------------------------
% O formato das figuras depende da forma como o arquivo de saída é gerado.
% As figuras inseridas na pasta Figures serão automaticamente reconhecidas sem
% a necessidade de inserir a extensão do arquivo.
%
% O pdfLaTEX (PDF) suporta figuras com as extensões: pdf, jpg, png e mps.
%
% -------------------------------------------------------------------------------
% Árvore do diretório a-project.tex
%  Diretório
%       \Classes        (requerido)
%       \Figures        (requerido) --------------------------------->
%       \Figures\PDF    (optional)
%       \Figures\JPG    (optional) Figures located within these
%       \Figures\PNG    (optional) folders are searched automatically
%       \Figures\MPS    (optional)  by the a-prj class.
%       \Figures\EPS    (optional)
%       \Figures\PS     (optional) <--------------------------------
%       \Tables         (requerido)
%       \Others         (requerido)
%       \Chapters       (requerido)
%       \Appendices     (optional)
%       \References     (requerido)
%
% -------------------------------------------------------------------------------
% PDF File resumo
\ifpdf
    \hypersetup{
    	backref,
        colorlinks  = true,
        pdftitle    = Modelo de documentação,
        pdfauthor   = {Marco Reis, marco.a.reis@gmail.com},
        pdfsubject  = Mestre em Engenharia,
        pdfcreator  = Subtitulo,
        pdfproducer = PDFLatex,
        pdfkeywords = {documentação, latex, dissertação, tese}}
 \fi
%
% -------------------------------------------------------------------------------
% Relação de pacotes opcionais utilizados
\usepackage[utf8]{inputenc}
\usepackage[brazil]{babel}
\usepackage{longtable}
\usepackage{dcolumn}
\usepackage{multirow}
\usepackage{lscape}
%\usepackage{graphicx}
\usepackage{rotating}
%\usepackage{float,subfigure}
%\usepackage{graphicx, subfigure}
\usepackage{cite}
\usepackage[left=3cm,top=3cm,right=2cm,bottom=2cm]{geometry}
\usepackage[alf]{abntex2cite}
\usepackage{ifpdf}
\usepackage{shadow}
\usepackage{wrapfig}
\usepackage[normalem]{ulem}
\usepackage{makeidx}
\usepackage{yfonts}
\usepackage{algorithm}
\usepackage{algorithmic}
\usepackage{lmodern}
\usepackage[T1]{fontenc}
\usepackage{indentfirst}
\usepackage{color}
\usepackage{microtype}
\usepackage{lipsum}
\usepackage{caption}
\usepackage{subcaption}
\usepackage{pdfpages}
\usepackage[pdftex]{hyperref}
%
\makeindex 
\setlength{\LTcapwidth}{\textwidth}
%
\newtheorem{theorem}{Teorema}
\newtheorem{definition}[theorem]{Definição}
%
% -------------------------------------------------------------------------------
% Configurações do pacote backref
\renewcommand{\backrefpagesname}{Citado na(s) página(s):~}
% Texto padrão antes do número das páginas
\renewcommand{\backref}{}
% Define os textos da citação
\renewcommand*{\backrefalt}[4]{
	\ifcase #1 %
		Nenhuma citação no texto.%
	\or
		Citado na página #2.%
	\else
		Citado #1 vezes nas páginas #2.%
	\fi
}
% 
% -------------------------------------------------------------------------------
% Início do documento raiz
\begin{document}
% Definição do título da página
    \university{Centro Universitário SENAI CIMATEC}
	\faculty{Pós-Graduação em róbotica e Sistemas autonomos}
	% \school{scola de...}
% 
    %\course{Engenharia Elétrica}
    % \typework{Centro de Competência em Robótica e Sistemas Autônomos}
% 
	%\course{Mestrado em Modelagem Computacional e Tecnologia Industrial}
	%\typework{Disserta\c{c}\~ao de mestrado}
	%\typework{Exame de Qualificação de Mestrado}
% 
	%\course{Engenharia Elétrica}
	%\typework{Tese de doutorado}
	%\typework{Exame de Qualificação de doutorado}
%
% -------------------------------------------------------------------------------
% Informações gerais
    \thesistitle{Relatório final do programa ``Novos Talentos - Robótica e Sistemas Autônomos''}
    \hidevolume
    \thesisvolume{Volume 1 of 1}
    \thesisauthor{Vinicius José Gomes de Araújo Felismino}
    \thesisauthorr{John Nash}
    \thesisauthorrr{James Clerk Maxwell}
    \thesisauthorrrr{Nikola Tesla}
    \thesisauthorrrrr{Sir Isaac Newton}
    %\thesisadvisor{Prof. Marco Reis, M.Eng.}
    %\hidecoadvisor
    %\thesiscoadvisor{Marco Reis}
    \thesisdegreetitle{Bacharel em Engenharia}
    \thesismonthyear{Dezembro de 2020}
% 
    \maketitlepage
%
% ----------------------------------------------------------------------------
% Inserir Folha de rosto, Nota de estilo, folha de assinaturas, dedicatoria
    \include{Others/FolhaRosto}
    %\include{Others/NotaEstilo}
    %\include{Others/FolhaAssinaturas}
    %\include{Others/dedicatoria}
    %\include{Others/agradecimentos}
%
% ----------------------------------------------------------------------------
% Resumo/abstract, sumário e siglas
    \begin{romanpagenumbers}
        \begin{thesisresumo}
  
  O presente relatório tem o objetivo de apresentar os trabalhos desenvolvidos durante o período vigente da bolsa de estudos em robótica e sistemas autônomos. Esta oportunidade surgiu através do programa ``Novos Talentos - Robótica e Sistemas Autônomos'' oferecido pelo SENAI-CIMATEC no ano de 2019 cujo o principal objetivo foi de desenvolver capital humano diferenciado no estado da Bahia, buscando o aumento da oferta de mão de obra qualificada nas competências supracitadas. Os trabalhos desenvolvidos durante o período do programa permitiram atribuir conhecimentos nas áreas de robótica móvel e manipuladores robóticos e, consequentemente, proporcionaram a formação de um Especialista em Robótica e Sistemas Autônomos.

\ \\

% use de três a cinco palavras-chave

\textbf{Palavras-chave}: Robótica, Sistemas Autônomos, Robótica Móvel, Manipuladores.

\end{thesisresumo}

        \begin{thesisabastract}
  The purpose of this report is to present the works developed during the outstanding period of the scholarship in robotics and autonomous systems. This opportunity arose through the program “Novos Talentos - Robótica e Sistemas Autônomos” offered by SENAI-CIMATEC in 2019 whose main objective was to develop differentiated human capital in the state of Bahia, seeking to increase the supply of qualified labor in the above-mentioned skills. The works developed during the program period allowed to assign knowledge in the areas of mobile robotics and robotic manipulators, consequently, provided the training of a Specialist in Robotics and Autonomous Systems.

\ \\

% use de tr�s a cinco palavras-chave

\textbf{Keywords}: Robotics, Autonomous Systems, Manipulators, Mobile robotics.

\end{thesisabastract}

        % Make list of contents, tables and figures
        \thesiscontents
        %Include other required section
        %\include{Others/abbreviation}
        %\include{Others/simbolos}
        %Switch the page numbering back to the default format
    \end{romanpagenumbers}
%
% ---------------------------------------------------------------------------
% Include thesis chapters
    \parskip=\baselineskip
    \chapter{Introdução}
\label{chap:intro}

O presente documento visa agrupar todos os conteúdos abordados e mostrar os resultados das atividades de pesquisa, criação e seleção de soluções para todos os desafios propostos pelos os orientadores e pesquisadores do Centro de Competência em Robótica e Sistemas Autônomos (CCRoSA) do SENAI-CIMATEC. Todas estas atividades ocorram durante o processo do curso de formação em robótica e sistemas autônomos, oferecido pela instituição, que teve inicio no dia 9 de novembro de 2019 e um prazo de duração de 1 ano.



%--------- NEW SECTION ----------------------
\section{Objetivos}
\label{sec:obj}

O objetivo geral deste relatório é relatar como o curso foi estruturado, ou seja, será apresentado todas as atividades e conhecimentos adiquiridos durante todo o periodo de duração da formação. 


\subsection{Objetivos Específicos}
\label{ssec:objesp}

O objetivo específico deste documento é agrupar todas as soluções para os diversos desafios proposto. 

%--------- NEW SECTION ----------------------
\section{Justificativa}
\label{sec:justi}

Com os crecentes avanços técnológicos, principalmente na área da robótica, a instituição SENAI-CIMATEC observou que era necessário mão de obra qualificada neste campo. Partindo desta premissa e adotando-a como seu principal objetivo lançou o programa de formação ``Novos Talentos - Robótica e Sistemas Autônomos'' para buscar o aumento da oferta de pessoal qualificado, em especial para as competências de robótica e sistemas Autônomos. 


%--------- NEW SECTION ----------------------
\section{Organização do documento}
\label{section:organizacao}

Este documento apresenta $4$ capítulos e está estruturado da seguinte forma:

\begin{itemize}

  \item \textbf{Capítulo \ref{chap:intro} - Introdução}: Contextualiza o âmbito, no qual o programa proposto está inserido. Apresenta, portanto, os objetivos, justificativa e como este programa foi estruturado;
  % \item \textbf{Capítulo \ref{chap:fundteor} - Fundamentação Teórica}: XXX;
  \item \textbf{Capítulo \ref{chap:mat} - Desenvolvimento}: Contextualiza cada problema proposto pelos os orientadores e pesquisadores do CCRoSA. Além disso, será demonstrado os materiais e métodos utilizados para a solução dos mesmos;
  \item \textbf{Capítulo \ref{chap:result} - Resultados}: Será exibido os principais resultados obtidos com a resolução de cada desafio proposto;
  \item \textbf{Capítulo \ref{chap:conc} - Conclusão}: Apresenta a conclusão geral do programa além de apresentar uma conclusão específica para cada desafio.

\end{itemize}

    % \include{Chapters/fundamentos}
    \chapter{Materiais e Métodos}
\label{chap:mat}

Neste capítulo será descrito cada desafio proposto pelos os orientadores e pesquisadores do CCRoSA durante o ano letivo e os materiais e métodos adotados para a concepção das soluções para cada um deles.

% Neste capítulo será descrito os materiais e os métodos adotados para a concepção das soluções para cada desafio proposto pelo os orientadores e pesquisadores do CCRoSA durante o ano letivo. 

\section{Desafio 2.0 - Concepção de um manipulador robótico.}
\label{sec:desafio2.0}

A princípio o objetivo deste desafio foi construir um manipulador robótico e através da identificação de marcadores visuais realizar a tarefa de acionamento de um painel elétrico. Devido aos efeitos da COVID-19, as atividades presenciais no laboratório tiveram que ser suspensas, com isso, afetou diretamente na construção física do manipulador. Portanto, a solução entregue para este desafio foi realizada apenas em ambiente simulado. 

Para concepção da solução foi formado um grupo que, além de mim, detinha mais três bolsistas do programa, entre eles, Jéssica Motta, Leonardo Mendes e Miguel Felipe. A modelagem 3D da estrutura física do manipulador foi feita no \textit{software OnShape}, o \textit{framework} de robótica e manipulação utilizado foi o \textit{Robot Operating System (ROS)} e \textit{MoveIt}, respectivamente. Por fim, a simulação da tarefa foi realizada no \textit{software Gazebo} com auxilio da ferramenta de visualização \textit{Rviz}.

\section{Desafio 2.5 - Simular a marcha e corrida de revezamento com o Darwin-OP}
\label{sec:desafio2.5}
    \include{Chapters/metodologia}
    \chapter{Resultados}
\label{chap:result}
Neste capítulo serão descritos os resultados que cada relatório desenvolvido para as possíveis soluções de cada desafio gerou durante o programa de formação. Além disso, será apresentado um trabalho extra que, impulsionado devido aos efeitos do COVID-19, foi desenvolvido por alguns bolsistas juntamente com os orientadores e pesquisadores do centro de competência além do seu potencial resultado. 

%--------- NEW SECTION ----------------------
\section{Resultado do resumo expandido ``Timon-HM''}
\label{sec:testu}

O relatório desenvolvido durante a concepção da solução do desafio 2.0, conforme visto na seção \ref{sec:desafio2.0}, gerou, além da participação, a submissão do resumo expandido ``MANIPULADOR ROBÓTICO- TIMON-HM'' dos autores Jéssica Motta, Leonardo Lima, Miguel Felipe e Vinicius Felismino no V Seminário de Pesquisa Científica e
Tecnológica (SAPCT) e IV Workshop de Integração e Capacitação em Processamento de Alto Desempenho (ICPAD). Os certificados de participação e submissão estão disponíveis nos anexos \ref{appen:sapct} e \ref{appen:resume_timon}, respectivamente.

\section{UGV MOCÓ: ROBÔ AUTÔNOMO TERRESTRE PARA DESINFECÇÃO DE AMBIENTES}
\label{sec:ugv}

Em virtude da pandemia criou-se a necessidade de desinfecção e higienização de ambientes para que pessoas possam transitar com mais segurança em locais cujo risco de contaminação é controlado. De modo a atender tal necessidade, propõe-se desenvolver uma plataforma robótica móvel capaz de percorrer ambientes de forma autônoma, empregando métodos
de desinfecção recomendados pela comunidade científica. 

Deste modo, o Apendice \ref{appen:ugv} apresenta a fase de concepção do projeto, detalhando as soluções atuais e o conceito proposto. Traz-se a especificação do sistema, modelo esquemático e a arquitetura geral do sistema. A equipe responsável pelo desenvolvimento deste projeto foi: Anderson Queiroz do Vale, Israel Cerqueira Motta Neto, Vinicius José Gomes de Araújo Felismino, Rebeca Tourinho Lima e Marco Antonio dos Reis.


\section{Artigo publicado PERSPECTIVES ON AUTONOMOUS UNMANNED GROUND
VEHICLES: A SURVEY}

Atualmente, os robôs móveis têm sido usados em diferentes ambientes e
aplicações, até mesmo para tarefas domésticas. O uso desses dispositivos pode aumentar a segurança e a produtividade, especialmente em operações industriais.

Partindo deste conceito e atrelado com a concepção da plataforma móvel,  visto na seção \ref{sec:ugv}, este artigo teve o objetivo de identificar gaps tecnológicos por meio de uma seleção atualizada
e por uma comparação de veículos terrestres não tripulados com rodas (UGVs). Com base em uma revisão sistemática e em uma classificação de robôs comercialmente disponíveis. Os resultados desta pesquisa pode ser visto no Apêndice \ref{appen:ugv_artigo}. 

Além disso, este artigo foi aceito para para publicação no
VI International Symposium on Innovation and Technology (SIINTEC), realizado no período de 21/10/2020 a 23/10/2020. O certificado de publicação pode ser visto no Apêndice \ref{appen:ugv_certificado}.
    \chapter{Conclusão}
\label{chap:conc}

O presente relatório buscou apresentar, além da metodologia utilizada, todas as atividades realizadas durante o período vigente do programa de formação. Com isso, permitiu adquirir conhecimentos nas áreas da robótica e sistema autônomos, consideradas como ramos tecnológicos multidisciplinar, principalmente no que se refere a manipuladores robóticos e robótica móvel.  

O programa de formação apresentou ferramentas importantes para o desenvolvimento de toda e qualquer pesquisa nestas áreas, dentre elas destacam-se: \textit{Trello}, \textit{Element}, \textit{GitHub}, \textit{RStudio}, \textit{Visual Code}. O \textit{trello} e o \textit{element} são duas ferramentas importantes no quesito de gestão de projetos pois a primeira, além de simples de operar e contar com uma versão gratuita, possibilita resolver situações de fluxo de atividades e da gestão de equipes no seu projeto por contar com uma série de integrações prontas com outros softwares, já a segunda possibilita uma comunicação segura entre os integrantes do projeto pois utiliza o método de criptografia das mensagens. O \textit{GitHub} é uma ferramenta com grande importância pois possibilita o versionamento do projeto. Por fim, o \textit{Visual Code} e \textit{RStudio} são duas ferramentas importantes para desenvolvimento de código em qualquer linguagem de programação, além de possuir diversas extensões e bibliotecas para facilitar no seu uso. É importante lembrar que o programa sempre buscou utilizar ferramentas de natureza \textit{open source}.

Além disso, todos os projetos de robótica foram desenvolvidos utilizando o \textit{framework} \textit{ROS}, pois forneceu as bibliotecas e ferramentas necessárias para auxiliar na concepção das funcionalidades do manipulador e da plataforma móvel. Além de que esta ferramenta possibilitou a simulação de todas as atividades proposta em cada desafio e com isso observar que a simulação pode ser totalmente diferente do mundo real.  

Durante o programa também foram desenvolvidas atividades que não foram citadas neste relatório mas que também foram importantes, entre elas, as apresentações individuais sobre artigos com temas propostos, liderança em projetos e alguns encontro para discutir sobre gestão de projetos. Todas estas permitiram criar e/ou aperfeiçoar habilidades nestas áreas.

De forma geral, o programa de formação ``Novos talentos - Robótica e Sistemas Autônomos'' oferecido pelo SENAI-CIMATEC em novembro de 2019 até novembro de 2020 permitiu por em prática todos os conceitos de robótica e sistemas autônomos vistos durante o ano letivo e isto só foi possível graças ao corpo docente e a infraestrutura do Centro de Competência em Robótica e Sistemas Autônomos (CCRoSA). Com isso podemos concluir que este programa, mesmo com as adversidades causadas pelo COVID-19, atendeu aos objetivos oferecidos em sua programação e assim proporcionou a formação de um Especialista em Robótica e Sistemas Autônomos. 





    % include more chapters ...
%
% ----------------------------------------------------------------------------
% Include thesis appendices
    \begin{thesisappendices}
        \chapter{Manipulador robótico TIMON-HM - Relatório parcial do projeto manipuladores inteligentes.}
        \label{appen:timon_hm_parcial}
        \includepdf[pages = {1-51}]{Appendices/Relatorio_parcial_Manipulador.pdf}

        \chapter{Avaliação do sistema de medição - TIMON-HM 2.5}
        \label{appen:timon_anova}
        \includepdf[pages={1-8}]{Appendices/timon_anova.pdf}

        \chapter{Manipulador robótico JeRoTIMON - Relatório final do projeto manipuladores inteligentes.}
        \label{appen:jerotimon}
        \includepdf[pages={1-134}]{Appendices/Relatorio_final_Manipulador.pdf}

        \chapter{Planejamento de experimentos (DOE) - Helicóptero de papel (TIMON-HM)}
        \label{appen:doe_timon}
        \includepdf[pages={1-18}]{Appendices/Relatorio_DOE.pdf}

        \chapter{Sistema para identificação e manipulação de objetos em ambientes externos utilizando plataforma móvel.}
        \label{appen:wbm}
        \includepdf[pages={1-115}]{Appendices/Relatorio_final_WarthogBombMission.pdf}

        \chapter{Certificado de participação do V Seminário de Pesquisa Científica e Tecnológica (SAPCT) e IV Workshop de Integração e Capacitação em Processamento de Alto Desempenho(ICPAD).}
        \label{appen:sapct}
        \includepdf[scale = 0.8, angle = 90]{Appendices/participacaosapct.pdf}

        \chapter{Certificado de submissão do resumo expandido - Manipulador Robótico - TIMON-HM no V Seminário de Pesquisa Científica e Tecnológica (SAPCT) e IV Workshop de Integração e Capacitação em Processamento de Alto Desempenho(ICPAD).}
        \label{appen:resume_timon}
        \includepdf[scale = 0.8,angle = 90, pagecommand={} ]{Appendices/CertificadoSAPCT.pdf}

        \chapter{UGV MOCÓ: ROBÔ AUTÔNOMO TERRESTRE PARADESINFECÇÃO DE AMBIENTES}
        \label{appen:ugv}
        \includepdf[pages={1-60}]{Appendices/Relatorio_parcial_UGV-M.pdf}

        \chapter{Artigo publicado PERSPECTIVES ON AUTONOMOUS UNMANNED GROUND VEHICLES: A SURVEY}
        \label{appen:ugv_artigo}
        \includepdf[pages={1-8}]{Appendices/sintec.pdf}

        \chapter{Certificado de publicação do artigo PERSPECTIVES ON AUTONOMOUS UNMANNED GROUND VEHICLES: A SURVEY}
        \label{appen:ugv_certificado}
        \includepdf[pages={1}, angle = 90, scale = 0.8]{Appendices/CertificadoVISIINTEC.pdf}
    \end{thesisappendices}
%
% ----------------------------------------------------------------------------
% Configurar as referencias bibliograficas
% 	\renewcommand\bibname{Referências}
%     \addcontentsline{toc}{chapter}{Referências}
%     \bibliography{References/referencias}
% %
% ----------------------------------------------------------------------------
% Finishing him
    \include{Others/ultimafolha}
\end{document}
%
% -------------------------------------------------------------------------------
% Aqui termina a formatação para o documento.
% In God We Trust. All Other Bring Data. 
%
% -------------------------------------------------------------------------------